\documentclass[12pt]{article}
\usepackage{ctex,amsmath,amsfonts,amssymb,bm,hyperref}
\author{付祈安}
\title{数学公式}
\begin{document}
\maketitle
\tableofcontents
\listoftables

\section{基础}
行内公式:
$ (x_1 + x_2)^2 = (x_1 - x_2)^2 + 4 x_1 x_2 $ \\
希腊字母:
\begin{equation*}
	\alpha, \beta, \gamma
\end{equation*}
\begin{displaymath}
	\delta, \Delta, \Psi, \Omega
\end{displaymath}
等号/不等号:
\[ =, \neq, \leq, \geq, \leqslant, \geqslant, \equiv \]
\[ \approx, \ll, \gg \]
分数:
\[ \frac{a}{b}, \frac ab \]
微积分:
\[ \int^a_b, \lim_{n \rightarrow \infty} \]
\[ \sum_{n=0}^{\infty}, \prod_\epsilon \]
其它:
\[ \bar{a}, \overline{a+b}, \underline{a+b} \]
\[ \vec{a}, \overrightarrow{AB} \]
\[ \underbrace{a_1+a_2+\ldots+a_n}_n \]
\[ \overbrace{a_1+a_2+\ldots+a_n}^n \]
\[ \binom{n}{k}, \mathrm{C}_n^k \]
\[ \stackrel{?}{=} \]
\[ \cdot, \cdots, \dots, \ldots \]
\[ \circ, \times \]
$ \frac ab\ {\displaystyle \frac ab} $
\section{字体}
\[ \mathbb{R}, \mathbf{B}, \boldmath{B} \]
\[ \mathrm{Hi},\ I\ have\ some\ \text{Text}. \]
\section{公式(equation)环境}
\begin{equation}
	\label{eq1}
	\left\{ \left[ \left( \frac{1}{1+x^2} \middle/ (1 + y) \right) \right] \right\}
	\quad \left. \frac{\mathrm{d}f}{\mathrm{d}x} \right|_{x = 0}
\end{equation}
使用星号*取消自动标号:
\begin{equation*}
	\int\!\!\!\int f(x, y) \; \mathrm{d} x \mathrm{d} y
	\quad \text{or} \quad \iint
\end{equation*}
\section{表格(table)}
\begin{table}[htbp]
	\begin{center}
		\begin{tabular}{|l|c|r|}
			\hline
			\multicolumn{2}{|c|}{Value} & third     \\ \hline
			1                           & 2     & 3 \\ \cline{1-1}
		\end{tabular}
		\caption{表格示例}
		\label{tab1}
	\end{center}
\end{table}
\section{矩阵}
\begin{displaymath}
	\mathbf{x}=
	\begin{pmatrix}
		x_{11} & x_{12} & \ldots \\
		x_{21} & x_{22} & \ldots \\
		\vdots & \vdots & \ddots
	\end{pmatrix}
\end{displaymath}
\section{多行公式}
\begin{align}
	a     & = b + c \\
	c + d & = e
\end{align}
\begin{equation}
	\begin{cases}
		\begin{aligned}
			a     & = b + c \\
			c + d & = e
		\end{aligned}
	\end{cases}
\end{equation}
\section{自定义命令}
\newcommand{\ud}{\mathrm{d}}
\newcommand{\dif}[2]{\frac{\ud {#1}}{\ud {#2}}}
\[ \dif fx \]
\end{document}
