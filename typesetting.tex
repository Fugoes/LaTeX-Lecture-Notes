\documentclass[12pt]{article}
\usepackage{ctex,graphicx,textcomp,amsmath,amsfonts}
\usepackage{amssymb,indentfirst,bm,subfigure,hyperref,xcolor}
\author{You Know Who}
\title{文字排版}
\begin{document}%注释
\maketitle
\newpage

\tableofcontents
\newpage

\part{结构}
\section{节}
\subsection{Apple Pen}
\subsubsection{Apple}
\subsubsection{Pen}
\subsection*{十分遗憾}
并没有 $\backslash$subsubsubsection

\section{换行、分段以及缩进}
第一行第一段。\\
下一行。\newline
\indent 再来一行。

分段\par
再来一段。

也可以这样换行。

\vspace{1em}

\noindent 没有缩进的新段落。

\section{脚注}
自信主动交流\footnote{阮东(2016)}

\part{文本}
\section{符号}\label{sec1}
\subsection{\LaTeX 中的标识符}
$\backslash$ \{ \^{} \_{} \^{} \} $\backslash$

\subsubsection{波浪线}
\~{} \qquad $\sim$  %\qquad 是一个比较长的空格

\subsubsection{横线}
X-men \\
page 13--67\\
yes --- or no?\\
0,1, and $-1$

\subsubsection{温度符号}
这里冬天平均$-30$\textcelsius ,夏天平均$50^{\circ}$F。

\subsubsection{日期}
\today

\subsubsection{强调}
你可以使用\textsl{斜体(实际上是楷体)}。\\
\emph{强调块里的强调是\emph{正常}文字。}

\subsubsection{货币}
\texteuro \  \textdollar

\subsubsection{音调类特殊符号}
Th\^o, na\"\i ve,\\
\= a \' a \v a \` a \\
ji\` an du\= o sh\' i gu\v ang

\subsubsection{英文引号}
``Please press the `x' key.''

\subsubsection{连字}
如果看不惯``shelfful'',可以写``shelf\mbox{}ful''。

\subsubsection{公然炫技}
\TeX \\
\LaTeXe \\
\AmS-\LaTeX \\

\subsubsection{字体与颜色}
{\tiny{tiny font}} \textrm{roman} \textsf{sans serif} \texttt{mono} \textbf{bold} \textit{italic} \textsc{small caps}

\textcolor{yellow}{黄色}的脸孔有\textcolor{red}{红色}的污泥

\section{环境}
\subsection{项目符号、编号、说明}
\begin{enumerate}
    \item 请根据个人口味混合各种环境:
        \begin{itemize}
            \item 可能看起来很怪。
            \item[-] 用横线。
        \end{itemize}
    \item 所以要牢记:放在列表里面的东西,
        \begin{description}
            \item[蠢的]不会变聪明;
            \item[聪明的]会变漂亮。
        \end{description}
\end{enumerate}

\subsection{对齐}
\begin{enumerate}
    \item 左对齐
        \begin{flushleft}
            这些文字都是\\ 左对齐的。

            \LaTeX{}不保证每行长度相同。
        \end{flushleft}
    \item 右对齐
        \begin{flushright}
            这些文字是\\ 右对齐的。
        \end{flushright}
    \item{居中}
        \begin{center}
            千里莺啼绿映红\\水村山郭酒旗风
        \end{center}
\end{enumerate}
\subsection{引用}
\begin{enumerate}
    \item 引用\\
        出于印刷要求,每行的长度要求为:
        \begin{quote}
            平均来说,每行不应超过66个半宽字符。
        \end{quote}
        这就是\LaTeX{}的页面有如此大的缺省页边距,而且报纸使用多列印刷的原因。
    \item 诗歌版式
        \begin{flushleft}
            \begin{verse}
                过完了这个月,我们打开门 \\
                一些花开在高高的树上 \\
                一些果结在深深的地下 \\
                \par \rightline{\emph{海子}《新娘》}
            \end{verse}
        \end{flushleft}
\end{enumerate}
\subsection{逐字输出}
\begin{verbatim}
    Hello world!
\end{verbatim}
Hello world!
\subsection{摘要}
\begin{abstract}
    摘要的摘要\label{abs}
\end{abstract}
\subsection{图片}
\begin{figure}[!htp]
    \centering
    \includegraphics[width=0.2\textwidth]{logo.jpg}
    \caption{Logo}\label{logo}
\end{figure} %插入图片,注意图片格式问题。
\subsection{交叉引用}
如下所示,可以引用到图形、表格、节或页面。\\
转到第\pageref{sec1}页的第\ref{sec1}节。\\
到第\ref{logo}个图形。
\end{document}
