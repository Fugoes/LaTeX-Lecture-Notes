\documentclass[12pt]{article}
\usepackage{xeCJK,graphicx,textcomp,syntonly,amsmath,amsfonts}
\usepackage{amssymb,etoolbox,indentfirst,bm,subfigure,hyperref}
\author{Xinyu Zheng \thanks{Based on lecture notes by Guanhao Huang, Zijia Chen, and Sirui Lu, and work by Oetiker et{} al.}}
\title{Typesetting}
\begin{document}%I'm notes.
\maketitle
\newpage

\tableofcontents
\newpage

\part{Structure}

\section{Section}
See as follows.
\subsection{Apple Pen}
\subsubsection{Apple}
\subsubsection{Pen}
\subsection*{A pity}
There's no `$\backslash$subsubsubsection'. So you can not see pen-pineapple apple pen here.

\section{New Line, New Paragraph and Indent}
This is the first line, first paragraph.\\
This is a new line.\newline
\indent This is another new line.

This is a new paragraph.\par
This is another new paragraph.

You can get a new line like this as well.

\noindent Haha, this is a new paragraph without indent! 

\section{Footnote}
Confident, initiative, and communication!\footnote{D. Ruan(2016)}

\part{Text}
\section{Marks}\label{sec1}
\subsection{Identifier In \LaTeX}
$\backslash$ \{ \^{} \_{} \^{} \} $\backslash$

\subsection{Some other marks}
\subsubsection{Tilde}
\~{} \qquad $\sim$  %\qquad is a relatively long blank.
\subsubsection{Dash}
X-men \\
page 13--67\\
yes --- or no?\\
0,1, and $-1$

\subsubsection{Degree Symbol}
It's $-30\textcelsius$ here in winter and some $50^{\circ}$F in summer.

\subsubsection{Date}
It's \today.

\subsubsection{Emphasize}
\emph{If you use emphasizing inside a piece of emphasizing text, then \LaTeX{} uses the \emph{normal} font for emphasizing.}\\
You can \textsl{lean} on me!

\subsubsection{Money}
\texteuro \  \textdollar

\subsubsection{Accent and Special Character}
H\^otel, na\"\i ve, \'el\`eve,\\
\= a \' a \v a \` a \\
sh\= an qi\' ong shu\v i j\`in

\subsubsection{Quotation}
``Please press the `x' key.''

\subsubsection{Ligature}
If you hate ``shelfful'', you can write it as ``shelf\mbox{}ful''.

\subsubsection{To Show Off}
\TeX \\
\LaTeXe \\
\AmS-\LaTeX \\

\section{Environment}
\subsection{Itemize, Enumerate, Description}
\begin{enumerate}
\item You can mix the list environment to your taste:
  \begin{itemize}
  \item But it might start to look silly.
  \item[-] With a dash.
  \end{itemize}
\item Therefore remember:
  \begin{description}
  \item[stupid] things will not become smart because they are in a list.
  \item[smart] things, though, can be presented beautifully in a list.
  \end{description}
\end{enumerate}

\subsection{Flush}
\begin{enumerate}
\item Flush left
  \begin{flushleft}
    This text is\\ left-aligned.

    \LaTeX{} is not trying to make each line the same length.
  \end{flushleft}
\item Flushright
  \begin{flushright}
    This text is right-\\aligned.
    \LaTeX{} is not trying to make each line the same length.
  \end{flushright}
\item{Center}
  \begin{center}
    At the centre\\of the earth
  \end{center}
\end{enumerate}
\subsection{Quotation}
\begin{enumerate}
\item quote\\
  A typographical rule of thumb for the line length is:
  \begin{quote}
    On average, no line should be longer than 66 characters.
  \end{quote}
  This is why \LaTeX{} pages have such large borders by default and also why multicolumn print is used in newspapers.
\item verse\\
  I know only one English poem by heart. It is about Humpty Dumty.
  \begin{flushleft}
    \begin{verse}
      Humty Dumty sat on a wall:\\
      Humty Dumty had a great fall.\\
      All the king's horse and all the King's man\\
      Couldn't put Humty Dumty together again.
    \end{verse}
  \end{flushleft}
\end{enumerate}
\subsection{Verbatim}
\begin{verbatim}
  Hello world!
\end{verbatim}
Hello world!
\subsection{Abstract}
\begin{abstract}
  The abstract's abstract.\label{abs}
\end{abstract}
\subsection{Figure}
\begin{figure}[!htp] 
  \centering
  \includegraphics[width=0.2\textwidth]{logo.jpg}
  \caption{Logo}\label{logo}
\end{figure} %插入图片,注意图片格式问题。
\subsection{Cross-references}
You can refer to figure, table, section, and page as follows.\\
Turn to Section \ref{sec1} on page \pageref{sec1}.\\
This is Figure \ref{logo}.
\end{document}
