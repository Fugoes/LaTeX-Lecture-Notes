\documentclass[12pt]{article}
\usepackage{ctex,graphicx,textcomp,syntonly,amsmath,amsfonts}
\usepackage{amssymb,etoolbox,indentfirst,bm,subfigure,hyperref}
\author{郑鑫宇 \thanks{Based on lecture notes by Guanhao Huang, Zijia Chen, and Sirui Lu, and work by Oetiker et{} al. 由王宇逸进行有选择地汉化。}}
\title{排版}
\begin{document}%注释
\maketitle
\newpage

\tableofcontents
\newpage

\part{结构}

\section{节(Section)}
见下:
\subsection{Apple Pen}
\subsubsection{Apple}
\subsubsection{Pen}
\subsection*{十分遗憾}
并没有`$\backslash$subsubsubsection',所以没有pen-pineapple apple pen。

\section{换行、分段以及缩进}
第一行第一段。\\
下一行。\newline
\indent 再来一行。

分段\par
再来一段。

也可以这样换行。

\noindent 没有缩进的新段落。

\section{脚注}
自信主动交流,然后找到……\footnote{阮东(2016)}

\part{文本}
\section{符号}\label{sec1}
\subsection{\LaTeX 中的标识符}
$\backslash$ \{ \^{} \_{} \^{} \} $\backslash$

\subsection{其它符号}
\subsubsection{波浪线}
\~{} \qquad $\sim$  %\qquad 是一个比较长的空格
\subsubsection{横线}
X-men \\
page 13--67\\
yes --- or no?\\
0,1, and $-1$

\subsubsection{温度符号}
这里冬天平均$-30$\textcelsius ,夏天平均$50^{\circ}$F。

\subsubsection{日期(英文)}
It's \today.

\subsubsection{强调(Emphasize)}
你可以使用\textsl{斜体(实际上是楷体)(italic)}。\\
\emph{强调块里的强调是\emph{正常}文字。}

\subsubsection{货币}
\texteuro \  \textdollar

\subsubsection{音调类特殊符号}
Th\^o, na\"\i ve,\\
\= a \' a \v a \` a \\
ji\` an du\= o sh\' i gu\v ang

\subsubsection{英文引号}
``Please press the `x' key.''

\subsubsection{连字}
如果看不惯``shelfful'',可以写``shelf\mbox{}ful''。

\subsubsection{公然炫技}
\TeX \\
\LaTeXe \\
\AmS-\LaTeX \\

\section{环境}
\subsection{项目符号、编号、说明}
\begin{enumerate}
	\item 请根据个人口味混合各种环境:
	\begin{itemize}
		\item 可能看起来很怪。
		\item[-] 用横线。
	\end{itemize}
	\item 所以要牢记:放在列表里面的东西,
	\begin{description}
    	\item[蠢的]不会变聪明;
    	\item[聪明的]会变漂亮。
	\end{description}
\end{enumerate}

\subsection{对齐}
\begin{enumerate}
	\item 左对齐
	\begin{flushleft}
		这些文字都是\\ 左对齐的。

		\LaTeX{}不保证每行长度相同。
	\end{flushleft}
	\item 右对齐
	\begin{flushright}
		这些文字是\\ 右对齐的。
	\end{flushright}
	\item{居中}
	\begin{center}
		千里莺啼绿映红\\水村山郭酒旗风
	\end{center}
\end{enumerate}
\subsection{引用}
\begin{enumerate}
	\item 引用\\
	出于印刷要求,每行的长度要求为:
	\begin{quote}
		平均来说,每行不应超过66个字符。
	\end{quote}
	这就是\LaTeX{}的页面有如此大的缺省页边距,而且报纸使用多列印刷的原因。
	\item 诗歌版式\\
	I know only one English poem by heart. It is about Humpty Dumty.
	\begin{flushleft}
		\begin{verse}
			Humty Dumty sat on a wall:\\
			Humty Dumty had a great fall.\\
			All the king's horse and all the King's man\\
			Couldn't put Humty Dumty together again.
		\end{verse}
	\end{flushleft}
\end{enumerate}
\subsection{逐字输出(Verbatim)}
\begin{verbatim}
  Hello world!
\end{verbatim}
Hello world!
\subsection{摘要(Abstract)}
\begin{abstract}
	摘要的摘要\label{abs}
\end{abstract}
\subsection{图形(Figure)}
\begin{figure}[!htp]
	\centering
	\includegraphics[width=0.2\textwidth]{logo.jpg}
	\caption{Logo}\label{logo}
\end{figure} %插入图片,注意图片格式问题。
\subsection{交叉引用}
如下所示,可以引用到图形、表格、节或页面。\\
转到第\pageref{sec1}页的第\ref{sec1}节。\\
到第\ref{logo}个图形。
\end{document}
